\section[system/beskrivelse]

Fælles for alle PC kølesystemerne i denne rapport er den første komponent der leder varmen væk fra processoren.

Kølesystemet for en computerprocessor i første indsats består af en varmeledning af varme fra processor til kølesystem.
Ofte faciliteres varmetransmissionen ved hjælp af en pasta, for at maximere kontaktflade til kølesystemet og for at modvirke galvanisk tæring.
I kildelisten er to eksempler på en pasta, med konduktivitets værdier på 0,8 og 12,5 W/m*K.
imidlertid er intentionen med kølepasta, at optimere overflade kontakten imellem kølegitter og processor og ikke skabe et seperat lag, hvor der kan dannes ekstra varme modstand. Så laget af kølepasta vil her blive negligeret. På en moderne processor med millioner af transistorer, er det set at processoren kan smelte ved store belastninger. 

Så kort sagt, er det system vi undersøger et system der er afgrænset af interfacet til processoren og af bortledning af varme fra det sidste komponent af kølesystemet.