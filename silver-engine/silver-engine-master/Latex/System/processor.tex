\section{Processor}

Den elektriske effekt P afsat i en CPU kan beskrives således :

 P = strømforbrug (I) * taktfrekvens (f) * kapacitans (C) .

 

I kølesystemet for en computerprocessor foregår den første varmetransmission som varmeledning fra processorens varmespreder til kølesystemet.

Ofte faciliteres denne varmetransmissionen af en kølepasta, som er med til at maksimere kontaktfladen til kølesystemet og som modvirker galvanisk tæring.

I kildelisten er inkluderet to eksempler på kølepasta, med konduktivitets værdier på hhv. 0,8 og 12,5 W/m*K.

Intentionen med kølepasta er at optimere kontakten mellem kølegitter og processor og undgå at der opstår små luftlommer, hvor der kan dannes ekstra varmemodstand. Dog vil laget af kølepasta her blive betragtet som liggende udenfor systemgrænsen.

De fleste kølegitre er lavet af aluminum, som har en varmekonduktivitet er på 229 W/m*K.

Vi antager et en dimensionel udbredelse af varme, stationært system med konstant varmestrøm. Vi antager ydermere konstant densitet og temperatur på den omgivende, tørre atmosfæriske luft.

Tykkelserne på godset i den retning varmen udbredes i er 0.5 mm for kølegitteret og regnes som en massiv væg, idet processoren regnes for at have en ens temperatur i hele sit areal.  Varmen forsimples til at udbredes i en retning og i et stationært system.  Med disse antagelser kan varmestrømmen igennem lamellerne beregnes.
Luften antages at flytte sig 0.01 m/s(c, hastighed)